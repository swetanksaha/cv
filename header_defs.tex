\newcommand*{\emailalt}[1]{\def\@emailalt{#1}}

\renewcommand*{\namefont}{\huge\mdseries\upshape}
\renewcommand*{\addressfont}{\normalsize\mdseries}

\renewcommand*{\mobilephonesymbol}{\faMobile~}
\renewcommand*{\emailsymbol}{\faEnvelope~}
\newcommand*{\scholarsocialsymbol} {{\faGraduationCap}~}
\renewcommand*{\linkedinsocialsymbol}{\faLinkedinSquare~}
\renewcommand*{\githubsocialsymbol}{\faGithub~}

% adds a social link to one's personal information (optional)
% usage: \social[<optional type>][<optional url>]{<account name>}
% where <optional type> should be either "linkedin", "xing", "twitter", "github", "gitlab" or "skype"
\collectionnew{cv_socials}
\NewDocumentCommand{\cvsocial}{O{}O{}m}{%
  \ifthenelse{\equal{#2}{}}%
    {%
      \ifthenelse{\equal{#1}{linkedin}}{\collectionadd[linkedin]{cv_socials}{\protect\link[#3]{https://www.linkedin.com/in/#3}}} {}%
      \ifthenelse{\equal{#1}{xing}}    {\collectionadd[xing]{cv_socials}    {\protect\link[#3]{https://www.xing.com/profile/#3}}}{}%
      \ifthenelse{\equal{#1}{twitter}} {\collectionadd[twitter]{cv_socials} {\protect\link[#3]{https://www.twitter.com/#3}}}     {}%
      \ifthenelse{\equal{#1}{github}}  {\collectionadd[github]{cv_socials}  {\protect\link[#3]{https://www.github.com/#3}}}      {}%
      \ifthenelse{\equal{#1}{gitlab}}  {\collectionadd[gitlab]{cv_socials}  {\protect\link[#3]{https://www.gitlab.com/#3}}}      {}%
      \ifthenelse{\equal{#1}{skype}}   {\collectionadd[skype]{cv_socials}   {#3}}                                            {}%
      \ifthenelse{\equal{#1}{scholar}} {\collectionadd[scholar]{cv_socials} {\protect\link[Google scholar]{https://scholar.google.com/citations?user=#3&hl=en}}}      {}%
    }
    {\collectionadd[#1]{cv_socials}{\protect\link[#3]{#2}}}}

% commands
\@initializecommand{\makeheaddetailssymbol}{%
    {~~~{\rmfamily\textbullet}~~~}}% the \rmfamily is required to force Latin Modern fonts when using sans serif, as OMS/lmss/m/n is not defined and gets substituted by OMS/cmsy/m/n
%   internal command to add an element to the footer
%   it collects the elements in a temporary box, and checks when to flush the box
\@initializebox{\makeheaddetailsbox}%
\@initializebox{\makeheaddetailstempbox}%
\@initializelength{\makeheaddetailswidth}%
\@initializelength{\makeheaddetailsboxwidth}%
\@initializeif{\if@firstmakeheaddetailselement}\@firstmakeheaddetailselementtrue%
%   adds an element to the makehead, separated by makeheadsymbol
%   usage: \addtomakehead[makeheadsymbol]{element}
\newcommand*{\addtomakeheaddetails}[2][\makeheaddetailssymbol]{% TODO: use \@initializecommand, which requires modifying its definition to handle mandatory and optional arguments
  \if@firstmakeheaddetailselement%
    \savebox{\makeheaddetailstempbox}{\usebox{\makeheaddetailsbox}#2}%
  \else%
    \savebox{\makeheaddetailstempbox}{\usebox{\makeheaddetailsbox}#1#2}\fi%
  \settowidth{\makeheaddetailsboxwidth}{\usebox{\makeheaddetailstempbox}}%
  \ifnum\makeheaddetailsboxwidth<\makeheaddetailswidth%
    \savebox{\makeheaddetailsbox}{\usebox{\makeheaddetailstempbox}}%
    \@firstmakeheaddetailselementfalse%
  \else%
    \flushmakeheaddetails\\%
    \savebox{\makeheaddetailsbox}{#2}%
    \savebox{\makeheaddetailstempbox}{#2}%
    \settowidth{\makeheaddetailsboxwidth}{\usebox{\makeheaddetailsbox}}%
    \@firstmakeheaddetailselementfalse\fi
}

%   internal command to flush the makehead
\@initializecommand{\flushmakeheaddetails}{%
  \strut\usebox{\makeheaddetailsbox}%
  \savebox{\makeheaddetailsbox}{}%
  \savebox{\makeheaddetailstempbox}{}%
  \setlength{\makeheaddetailsboxwidth}{0pt}}

\@initializecommand{\makehead}{%
  \setlength{\makeheaddetailswidth}{0.8\textwidth}%
  \hfil%
  \parbox{\makeheaddetailswidth}{%
    \centering%
    % name and title
    \namestyle{\@firstname~\@lastname}%
    \ifthenelse{\equal{\@title}{}}{}{\titlestyle{~|~\@title}}\\% \isundefined doesn't work on \@title, as LaTeX itself defines \@title (before it possibly gets redefined by \title)
    % optional detailed information
    \if@details{%
        \addressfont%
        \ifthenelse{\isundefined{\@addressstreet}}{}{\addtomakeheaddetails{\addresssymbol\@addressstreet}%
        \ifthenelse{\equal{\@addresscity}{}}{}{\addtomakeheaddetails[~--~]{\@addresscity}}% if \addresstreet is defined, \addresscity and \addresscountry will always be defined but could be empty
        \ifthenelse{\equal{\@addresscountry}{}}{}{\addtomakeheaddetails[~--~]{\@addresscountry}}%
        \flushmakeheaddetails\@firstmakeheaddetailselementtrue\\\null}%
        \collectionloop{phones}{% the key holds the phone type (=symbol command prefix), the item holds the number
            \addtomakeheaddetails{\csname\collectionloopkey phonesymbol\endcsname\collectionloopitem}}%
        \ifthenelse{\isundefined{\@email}}{}{\addtomakeheaddetails{\emailsymbol\emaillink{\@email}}}%
        \ifthenelse{\isundefined{\@emailalt}}{}{\addtomakeheaddetails{\emailsymbol\emaillink{\@emailalt}}}%
        \flushmakeheaddetails\\
        \@firstmakeheaddetailselementtrue
        \ifthenelse{\isundefined{\@homepage}}{}{\addtomakeheaddetails{\homepagesymbol\link{\@homepage}}}%
        \collectionloop{cv_socials}{% the key holds the social type (=symbol command prefix), the item holds the link
            \addtomakeheaddetails{\csname\collectionloopkey socialsymbol\endcsname\collectionloopitem}}%
        \ifthenelse{\isundefined{\@extrainfo}}{}{\addtomakeheaddetails{\@extrainfo}}%
    \flushmakeheaddetails}\fi}\\[2.5em]}% need to force a \par after this to avoid weird spacing bug at the first section if no blank line is left after \makehead

\renewcommand*{\makecvhead}{%
  % recompute lengths (in case we are switching from letter to resume, or vice versa)
  \recomputecvlengths%
  \makehead
  \par}% to avoid weird spacing bug at the first section if no blank line is left after \makecvhead
