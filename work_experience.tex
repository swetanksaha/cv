\cventry{\textit{Summer 2017}\newline
\texttt{C}\newline
\texttt{Python}\newline
\texttt{OpenWRT}\newline
}{Research Associate (Intern)}{\textbf{Hewlett Packard Enterprise (HPE) Labs}}{Palo Alto, CA}{US}{
\underline{LTE-Unlicensed/WiFi Co-existence in 5 GHz for WiFi Access Points (APs)}
\vspace{0.025in}
\begin{itemize}
  \item Designed \textit{DeMiLTE}, the first WiFi (802.11ac)-based system for enterprise APs to detect, quantify, and react to LTE-U/LAA interference in real time, without requiring additional AP hardware
  \item Implemented the system inside the AP firmware making it light-weight and fully 802.11ac-standard compliant
  \item Improved AP downlink throughput by up to \textbf{110\%}, without requiring any client modifications
  \item Published 1 top-tier conference paper \hyperref[C3]{[C3]} and filed 1 patent \hyperref[P1]{[P1]}
\end{itemize}
}

\vspace{0.025in}
\cventry{\textit{Summer 2016}\newline
\texttt{C}\newline
\texttt{Python}\newline
\texttt{Drivers}\newline
\texttt{LEDE}\newline
}{Research Intern}{\textbf{IMDEA Networks Institute}}{Madrid}{Spain}{
\underline{Millimeter-wave Networking}
\vspace{0.025in}
\begin{itemize}
  \item Analyzed performance bottlenecks in next-generation of WiFi: 60 GHz (802.11ad)-based indoor WLANs
  \item Modified the Linux 802.11ad wireless device driver \texttt{(wil6210)} to export PHY/MAC information to userspace and allow userspace control over PHY parameters, like beam direction \link[\faGithub]{https://github.com/swetanksaha/wil6210}
  \item Undertook an extensive measurement study with the instrumented APs and laptops to study 802.11ad links.
  \item Highlighted novel challenges and practical aspects like coverage and AP deployment, previously unreported
  \item Published the results of the study in a top-tier conference \hyperref[C9]{[C9]}
\end{itemize}
}

\vspace{0.025in}
\cventry{\textit{Summer 2013}\newline
\texttt{Python}\newline
\texttt{Java}\newline
\texttt{Android}\newline
\texttt{Django}\newline
}{Software Developer}{\textbf{Google Summer of Code 2013}}{Google}{}{
\underline{\link[Funf]{https://www.funf.org/}: Open sensing and data collection framework for Android (acquired by Google)} \link[\faGithub]{https://github.com/funf-org/funf-inabox}
\vspace{0.025in}
\begin{itemize}
  \item Improved Funf-in-a-box (FIAB), a service for users to build custom data collection app with zero programming
  \item Ported the entire FIAB service from an always-on architecture running on a single EC2 server to on-demand VM instantiation on Google cloud (in <1 month), significantly reducing costs \& increasing performance. \link[\faExternalLink]{https://opensource.googleblog.com/2013/08/who-is-new-in-google-summer-of-code_30.html}
  \item Added support for configuring and deploying custom surveys and capturing additional user input
\end{itemize}
}

\vspace{0.025in}
\cventry{\textit{Summer 2012}\newline
\texttt{Android}\newline
\texttt{Ubuntu Juju}\newline}{Research Intern}{\textbf{Airbus India, EADS Innovation Works}}{Bangalore}{India}{
\underline{Image based localization techniques}
\vspace{0.025in}
\begin{itemize}
  \item Built an indoor localization service that uses image features to estimate location of an Android device
  \item Implemented it as a distributed system that offloads image processing to a remote server
  \item Packaged and deployed the system as a Juju charm for easy production orchestration
\end{itemize}
}

% \cventry{\textit{2013--Present}}{Research Assistant (RA) | Teaching Assistant (TA)}{\textbf{University at Buffalo}, SUNY}{NY}{US}{}

%\cventry{\textit{2013--Present}}{Teaching Assistant}{University at Buffalo, SUNY}{NY}{US}{}
%Designed and developed assignments and automated grader for the Modern Networking Concepts course\newline\httplink[https://cse4589.github.io/]{https://cse4589.github.io/} [\link[Github Source]{https://github.com/cse4589}]
