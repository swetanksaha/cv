\cventry{\textit{Summer 2017}\newline
\texttt{C}\newline
\texttt{Python}\newline
\texttt{OpenWRT}\newline
}{Research Associate (Intern)}{\textbf{Hewlett Packard Enterprise (HPE) Labs}}{Palo Alto, CA}{US}{
\underline{LTE-Unlicensed/WiFi Co-existence in 5 GHz}
\begin{itemize}
  \item Designed a WiFi (802.11ac)-based mechanism - \underline{DeMiLTE} for enterprise APs to combat LTE-U/LAA interference.
  \item DeMiLTE enables WiFi APs to detect, quantify and react to LTE interference in real time.
  \item DeMiLTE is light-weight, fully standard-compliant and requires no changes to LTE PHY/MAC.
  \item Implemented as AP-firmware, DeMiLTE improves performance by up to \textbf{110\%}, without modifying clients.
  \item Published 1 top-tier conference paper and filed 1 patent.
\end{itemize}
}

\cventry{\textit{Summer 2016}\newline
\texttt{C}\newline
\texttt{Python}\newline
\texttt{Drivers}\newline
\texttt{LEDE}\newline
}{Research Intern}{\textbf{IMDEA Networks Institute}}{Madrid}{Spain}{
\underline{Millimeter-wave Networking}
\begin{itemize}
  \item Analysis and protocol design for next-generation of WiFi: 60 GHz (802.11ad)-based indoor WLANs.
  \item Instrumented the linux wireless device driver (wil6210) to export PHY/MAC information to userspace.
  \item Modifed the Linux wil6210 driver to allow user control over PHY parameters, like beam direction, in real time.
  \item Undertook an extensive measurement study with the instrumented APs and laptops to study 802.11ad links.
  \item Study highlighted novel challenges and provided insights for practical aspects like coverage and AP deployment.
  \item Published the results of the study in a top-tier conference.
\end{itemize}
}

\cventry{\textit{Summer 2013}\newline
\texttt{Python}\newline
\texttt{Java}\newline
\texttt{Android}\newline
\texttt{Django}\newline}{Software Developer}{\textbf{Google Summer of Code 2013}}{Google}{}{
\underline{\link[Funf]{https://www.funf.org/}: Open sensing and data collection framework for Android (acquired by Google)}
\begin{itemize}
  \item Worked on Funf-in-a-box (FIAB), a service for users to build custom data collection app with zero programming.
  \item Ported the entire FIAB service from a single Amazon server (EC2) to Google Cloud Services (GCE) in 1 month.
  \item Switch from always-on architecture on EC2 to on-demand on GCE greatly reduced costs/increased performance.
  \item Added support for configuring and deploying custom surveys and capturing additional user input.
\end{itemize}
}

\cventry{\textit{Summer 2012}\newline
\texttt{Android}\newline
\texttt{Ubuntu Juju}\newline}{Research Intern}{\textbf{Airbus India, EADS Innovation Works}}{Bangalore}{India}{
\underline{Image based localization techniques}
\begin{itemize}
  \item Built an indoor localization service that uses image features to estimate position of an Android device.
  \item Implemented as a distributed system offloading image processing to a remote system accessed through an API.
  \item Packaged and deployed the system as a Juju charm for easy deployment and scalability.
\end{itemize}
}

\cventry{\textit{2013--Present}}{Research Assistant}{University at Buffalo, SUNY}{NY}{US}{}

\cventry{\textit{2013--Present}}{Teaching Assistant}{University at Buffalo, SUNY}{NY}{US}{}
%Designed and developed assignments and automated grader for the Modern Networking Concepts course\newline\httplink[https://cse4589.github.io/]{https://cse4589.github.io/} [\link[Github Source]{https://github.com/cse4589}]
