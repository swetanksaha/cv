\cventry{\texttt{C}\newline
\texttt{Python}\newline
\texttt{TCP}\newline
\texttt{MPTCP}\newline
\texttt{tc}\newline
\texttt{Kernel}\newline
}{Multipath TCP (MPTCP) for Dual-band WiGig (802.11ad)+WiFi (802.11ac) networks}{}{}{}{
\begin{itemize}
  \item Leveraged MPTCP to engage two network interfaces \textit{simultaneously} achieving throughputs of up to $\sim$2.2 Gbps
  \item Instrumented MPTCP (Linux 4.x) using \texttt{kernel probes} to monitor over 32 parameters in real-time
  \item Developed kernel tools to monitor TCP/MPTCP send and recv queues tracking ingress and egress for each byte
  \item Designed \textit{AMuSe}, a novel MPTCP scheduler to enable dynamic packet assignment based on network conditions
  \item Implemented it as a Linux kernel module improving overall throughput by up to \textbf{2.5x} under diverse scenarios
  % \item Evaluated AMuSe extensively which showed up to \textbf{2.5x} throughput improvement over the default scheduler
  \item Submitted a top-tier conference paper (under review)
\end{itemize}
}

\vspace{0.025in}
\cventry{\texttt{Python}\newline
\texttt{C}\newline
\texttt{Drivers}\newline
\texttt{Android}\newline
}{Power-Performance Tradeoffs for Mobile Devices in Next Generation WiFi Networks}{}{}{}{
\begin{itemize}
  \item Developed tools that allow for large-scale automated performance and power measurement of mobile devices
  \item Enabled data collection \& sync across heterogeneous systems: Linux (Wireless AP), Android, and Power monitor
  \item Modified the Linux wireless drivers (\texttt{ath9k}, \texttt{ath10k}) to expose userspace control of several PHY/MAC parameters.
  \item Analyzed the power-performance tradeoff of both the NIC and CPU for uplink/downlink data transfer
  \item Improved the accuracy of the state-of-the-art power model by up to \textbf{40\%}
  \item Pubished 1 top-tier conference paper \hyperref[C2]{[C2]}
\end{itemize}
}

\vspace{0.025in}
\cventry{\texttt{Labview}\newline
\texttt{Python}\newline
}{X60: A highly Re-configurable Multi-Gigabit Testbed for 60 GHz research}{}{}{}{
\begin{itemize}
  \item Set up the first ever software-defined \textit{X60}, a 60 GHz testbed that offers configurability at PHY/MAC/Network layers,
  supports phased antenna arrays and 2 GHz baseband bandwidth
  \item Added several components (e.g., AGC) to existing NI codebase to enable realistic measurements
  \item Developed set of tools to automate the measurement cycle reducing time from hours to several minutes
  \item Exposed user controls to enable measurements by external researchers (used by Rice University \& UT Austin)
  \item Related publications: \hyperref[J2]{[J2]}, \hyperref[W4]{[W4]}
\end{itemize}
}

\begin{comment}
\vspace{0.025in}
\cventry{\texttt{C}\newline
\texttt{Python}\newline
\texttt{expect}\newline
\texttt{select}\newline
\texttt{Socket API}\newline
}{Programming Assignments and Automated Grader for the Computer Networking course}{}{}{\link[\faGithub]{https://cse4589.github.io}}{
\begin{itemize}
  \item Designed and developed a set of three networking programming assignments (PAs), now standard in the CSE department at UB for the networking course.
  \item Developed an auto-grader as a distributed application that fully automates the entire process of packaging, uploading and grading student submissions over a real network of five servers.
  \item PA 1: Text Chat \& File Transfer Application \link[\faExternalLink]{https://cse4589.github.io/pa1/} (Source: \link[\faGithub]{https://github.com/swetanksaha/cse4589-pa1})
        \begin{itemize}
          \item Introduces basic socket programming, client-server model, \texttt{select} syscall to multiplex network and stdin I/O, and byte-stream model of TCP (vs. UDP's packet-based)
          %\item Bonus file-transfer components allows students to compare file I/O vs. network I/O.
        \end{itemize}
  \item PA 2: Reliable Transport Protocols \link[\faExternalLink]{https://cse4589.github.io/pa2/} (Source: \link[\faGithub]{https://github.com/swetanksaha/cse4589-pa2})
        \begin{itemize}
          \item Requires implementation of three reliable data transport protocols: Alternating bit, Go-Back-N \& Selective repeat, and multiple alarms with a single simulator timer
        \end{itemize}
  \item PA 3: Software Defined Routing \link[\faExternalLink]{https://cse4589.github.io/pa3/} (Source: \link[\faGithub]{https://github.com/swetanksaha/cse4589-pa3})
        \begin{itemize}
          \item Involves implementation of a simplified version of the distance vector routing protocol, introducing the concept of control and data planes asking them to implement both functionalities
          \item Requires contruction of raw packets in code using techniques akin to the Linux kernel (Sample code: \link[\faGithub]{https://github.com/swetanksaha/cse4589-pa3/blob/master/SampleCode/swetankk/src/control_header_lib.c})
        \end{itemize}
\end{itemize}
}
\end{comment}
