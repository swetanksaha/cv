\cventry{\texttt{C}\newline
\texttt{Python}\newline
\texttt{TCP}\newline
\texttt{MPTCP}\newline
\texttt{tc}\newline
\texttt{Kernel}\newline
}{Multipath TCP (MPTCP) for Dual-band WiGig+WiFi networks}{NSF Funded}{}{}{
\begin{itemize}
  \item Leveraged MPTCP to engage two network interfaces simultaneously to acheive throughput of $\sim$2.2 Gbps
  \item Instrumented MPTCP (Linux 4.x) using \texttt{kernel probes} to monitor over 32 parameters in real-time
  %\item Instrumented TCP/MPTCP send and recv queues to track input and output timestamp for each byte.
  \item Designed \underline{AMuSe}, a new MPTCP packet scheduler, that improved overall throughput by 2.5x
  \item Implemented it as a kernel module replacing the default MPTCP scheduler
  % \item Evaluated AMuSe extensively which showed up to \textbf{2.5x} throughput improvement over the default scheduler
  \item Submitted a top-tier conference paper (under review)
\end{itemize}
}

\cventry{\texttt{Python}\newline
\texttt{C}\newline
\texttt{Drivers}\newline
\texttt{Android}\newline
}{Power-Performance Tradeoffs for Mobile Devices in Next Generation WiFi Networks}{NSF Funded}{}{}{
\begin{itemize}
  \item Developed a set of tools to enable large-scale performance and power measurement of mobile devices involving data collection \& sync across heterogeneous systems: Linux (Wireless AP), Android, and Power monitor
  \item Modified the Linux wireless drivers (\texttt{ath9k}, \texttt{ath10k}) to expose userspace control of several PHY/MAC parameters.
  \item Analyzed the power-performance tradeoff of both the NIC and CPU for uplink/downlink data transfer.
  \item Improved the accuracy of the state-of-the-art power model by up to \textbf{40\%}.
  \item Pubished 1 top-tier conference paper \hyperref[C1]{[C1]}.
\end{itemize}
}

\cventry{\texttt{Labview}\newline
\texttt{Python}\newline
}{X60: A highly Re-configurable Multi-Gigabit Testbed for 60 GHz research}{NSF Funded}{}{}{
\begin{itemize}
  \item Set up the first ever software-defined 60 GHz testbed that offers configurability at PHY/MAC/Network layers.
  \item Added several components (e.g., AGC) to existing NI codebase to enable realistic measurements.
  \item Developed set of tools to automate the measurement cycle reducing time from hours to several minutes.
  \item Exposed user controls to enable measurements by external researchers (used by Rice University \& UT Austin).
  \item Published 1 journal paper \hyperref[J1]{[J1]}.
\end{itemize}
}

\cventry{\texttt{C}\newline
\texttt{Python}\newline
\texttt{expect}\newline
\texttt{select}\newline
\texttt{Socket API}\newline
}{Programming Assignments and Automated Grader for the Computer Networking course}{}{}{\link[\faGithub]{https://cse4589.github.io}}{
\begin{itemize}
  \item Designed and developed a set of three networking programming assignments (PAs), now standard in the CSE department at UB for the networking course.
  \item Developed the auto-grader as a distributed application that fully automates the entire process of packaging, uploading and grading student submissions over a real network of five servers.
  \item PA 1: Text Chat \& File Transfer Application \link[\faExternalLink]{https://cse4589.github.io/pa1/} (Source: \link[\faGithub]{https://github.com/swetanksaha/cse4589-pa1})
        \begin{itemize}
          \item Introduces basic socket programming, client-server model and use of \texttt{select} syscall.
          \item Test cases check understanding of the byte-stream model of TCP (vs. UDP's packet-based).
          %\item Bonus file-transfer components allows students to compare file I/O vs. network I/O.
        \end{itemize}
  \item PA 2: Reliable Transport Protocols \link[\faExternalLink]{https://cse4589.github.io/pa2/} (Source: \link[\faGithub]{https://github.com/swetanksaha/cse4589-pa2})
        \begin{itemize}
          \item Uses a simulated environment to ask students to implement three reliable data transport protocols.
          \item Requires implementation of multiple alarms with a single simulator timer.
          \item Modified an existing simulator to add sanity checks and internal support for different test cases.
        \end{itemize}
  \item PA 3: Software Defined Routing \link[\faExternalLink]{https://cse4589.github.io/pa3/} (Source: \link[\faGithub]{https://github.com/swetanksaha/cse4589-pa3})
        \begin{itemize}
          \item Involves implementation of a simplified version of the distance vector routing protocol (using UDP).
          \item Introduces students to the concept of control and data planes asking them to implement both functionalities.
          \item Requires contruction of raw packets in code using techniques akin to the Linux kernel (Sample code: \link[\faGithub]{https://github.com/swetanksaha/cse4589-pa3/blob/master/SampleCode/swetankk/src/control_header_lib.c}).
        \end{itemize}
\end{itemize}
}
