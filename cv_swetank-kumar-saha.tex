\documentclass[colorlinks,urlcolor=blue,10pt,letterpaper,sans]{moderncv}

\moderncvstyle{classic}
\moderncvcolor{blue}

\usepackage[margin=0.75in]{geometry}
\usepackage{fontawesome}
\usepackage{comment}

%\setlength{\hintscolumnwidth}{3cm} % Uncomment to change the width of the dates column
%\setlength{\makecvtitlenamewidth}{9cm} % For the 'classic' style, uncomment to adjust the width of the space allocated to your name

%----------------------------------------------------------------------------------------
%	Fixes: Swetank Kumar Saha
%----------------------------------------------------------------------------------------
\definecolor{color1}{rgb}{0,0,1.0}% dark blue
\makeatletter
\newcommand*{\emailalt}[1]{\def\@emailalt{#1}}

\renewcommand*{\namefont}{\huge\mdseries\upshape}
\renewcommand*{\addressfont}{\normalsize\mdseries}

\renewcommand*{\mobilephonesymbol}{\faMobile~}
\renewcommand*{\emailsymbol}{\faEnvelope~}
\newcommand*{\scholarsocialsymbol} {{\faGraduationCap}~}
\renewcommand*{\linkedinsocialsymbol}{\faLinkedinSquare~}
\renewcommand*{\githubsocialsymbol}{\faGithub~}

% adds a social link to one's personal information (optional)
% usage: \social[<optional type>][<optional url>]{<account name>}
% where <optional type> should be either "linkedin", "xing", "twitter", "github", "gitlab" or "skype"
\collectionnew{cv_socials}
\NewDocumentCommand{\cvsocial}{O{}O{}m}{%
  \ifthenelse{\equal{#2}{}}%
    {%
      \ifthenelse{\equal{#1}{linkedin}}{\collectionadd[linkedin]{cv_socials}{\protect\link[#3]{https://www.linkedin.com/in/#3}}} {}%
      \ifthenelse{\equal{#1}{xing}}    {\collectionadd[xing]{cv_socials}    {\protect\link[#3]{https://www.xing.com/profile/#3}}}{}%
      \ifthenelse{\equal{#1}{twitter}} {\collectionadd[twitter]{cv_socials} {\protect\link[#3]{https://www.twitter.com/#3}}}     {}%
      \ifthenelse{\equal{#1}{github}}  {\collectionadd[github]{cv_socials}  {\protect\link[#3]{https://www.github.com/#3}}}      {}%
      \ifthenelse{\equal{#1}{gitlab}}  {\collectionadd[gitlab]{cv_socials}  {\protect\link[#3]{https://www.gitlab.com/#3}}}      {}%
      \ifthenelse{\equal{#1}{skype}}   {\collectionadd[skype]{cv_socials}   {#3}}                                            {}%
      \ifthenelse{\equal{#1}{scholar}} {\collectionadd[scholar]{cv_socials} {\protect\link[Google scholar]{https://scholar.google.com/citations?user=#3&hl=en}}}      {}%
    }
    {\collectionadd[#1]{cv_socials}{\protect\link[#3]{#2}}}}

% commands
\@initializecommand{\makeheaddetailssymbol}{%
    {~~~{\rmfamily\textbullet}~~~}}% the \rmfamily is required to force Latin Modern fonts when using sans serif, as OMS/lmss/m/n is not defined and gets substituted by OMS/cmsy/m/n
%   internal command to add an element to the footer
%   it collects the elements in a temporary box, and checks when to flush the box
\@initializebox{\makeheaddetailsbox}%
\@initializebox{\makeheaddetailstempbox}%
\@initializelength{\makeheaddetailswidth}%
\@initializelength{\makeheaddetailsboxwidth}%
\@initializeif{\if@firstmakeheaddetailselement}\@firstmakeheaddetailselementtrue%
%   adds an element to the makehead, separated by makeheadsymbol
%   usage: \addtomakehead[makeheadsymbol]{element}
\newcommand*{\addtomakeheaddetails}[2][\makeheaddetailssymbol]{% TODO: use \@initializecommand, which requires modifying its definition to handle mandatory and optional arguments
  \if@firstmakeheaddetailselement%
    \savebox{\makeheaddetailstempbox}{\usebox{\makeheaddetailsbox}#2}%
  \else%
    \savebox{\makeheaddetailstempbox}{\usebox{\makeheaddetailsbox}#1#2}\fi%
  \settowidth{\makeheaddetailsboxwidth}{\usebox{\makeheaddetailstempbox}}%
  \ifnum\makeheaddetailsboxwidth<\makeheaddetailswidth%
    \savebox{\makeheaddetailsbox}{\usebox{\makeheaddetailstempbox}}%
    \@firstmakeheaddetailselementfalse%
  \else%
    \flushmakeheaddetails\\%
    \savebox{\makeheaddetailsbox}{#2}%
    \savebox{\makeheaddetailstempbox}{#2}%
    \settowidth{\makeheaddetailsboxwidth}{\usebox{\makeheaddetailsbox}}%
    \@firstmakeheaddetailselementfalse\fi
}

%   internal command to flush the makehead
\@initializecommand{\flushmakeheaddetails}{%
  \strut\usebox{\makeheaddetailsbox}%
  \savebox{\makeheaddetailsbox}{}%
  \savebox{\makeheaddetailstempbox}{}%
  \setlength{\makeheaddetailsboxwidth}{0pt}}

\@initializecommand{\makehead}{%
  \setlength{\makeheaddetailswidth}{0.8\textwidth}%
  \hfil%
  \parbox{\makeheaddetailswidth}{%
    \centering%
    % name and title
    \namestyle{\@firstname~\@lastname}%
    \ifthenelse{\equal{\@title}{}}{}{\titlestyle{~|~\@title}}\\% \isundefined doesn't work on \@title, as LaTeX itself defines \@title (before it possibly gets redefined by \title)
    % optional detailed information
    \if@details{%
        \addressfont%
        \ifthenelse{\isundefined{\@addressstreet}}{}{\addtomakeheaddetails{\addresssymbol\@addressstreet}%
        \ifthenelse{\equal{\@addresscity}{}}{}{\addtomakeheaddetails[~--~]{\@addresscity}}% if \addresstreet is defined, \addresscity and \addresscountry will always be defined but could be empty
        \ifthenelse{\equal{\@addresscountry}{}}{}{\addtomakeheaddetails[~--~]{\@addresscountry}}%
        \flushmakeheaddetails\@firstmakeheaddetailselementtrue\\\null}%
        \collectionloop{phones}{% the key holds the phone type (=symbol command prefix), the item holds the number
            \addtomakeheaddetails{\csname\collectionloopkey phonesymbol\endcsname\collectionloopitem}}%
        \ifthenelse{\isundefined{\@email}}{}{\addtomakeheaddetails{\emailsymbol\emaillink{\@email}}}%
        \ifthenelse{\isundefined{\@emailalt}}{}{\addtomakeheaddetails{\emailsymbol\emaillink{\@emailalt}}}%
        \flushmakeheaddetails\\
        \@firstmakeheaddetailselementtrue
        \ifthenelse{\isundefined{\@homepage}}{}{\addtomakeheaddetails{\homepagesymbol\link{\@homepage}}}%
        \collectionloop{cv_socials}{% the key holds the social type (=symbol command prefix), the item holds the link
            \addtomakeheaddetails{\csname\collectionloopkey socialsymbol\endcsname\collectionloopitem}}%
        \ifthenelse{\isundefined{\@extrainfo}}{}{\addtomakeheaddetails{\@extrainfo}}%
    \flushmakeheaddetails}\fi}\\[2.5em]}% need to force a \par after this to avoid weird spacing bug at the first section if no blank line is left after \makehead

\renewcommand*{\makecvhead}{%
  % recompute lengths (in case we are switching from letter to resume, or vice versa)
  \recomputecvlengths%
  \makehead
  \par}% to avoid weird spacing bug at the first section if no blank line is left after \makecvhead


\newcommand*{\pub}[6]{
\label{#1}\cvitem{#1}{\textbf{#2}\newline
           {\small \textit{#3} (\textbf{#4}) \textit{#5} \ifthenelse{\equal{#6}{}}{}{\link[\faExternalLink]{#6}} }}
}

\makeatother

\firstname{\textbf{Swetank Kumar}} % Your first name
\familyname{\textbf{Saha}} % Your last name

% All information in this block is optional, comment out any lines you don't need
% \title{{\normalfont \large Ph.D. Candidate\newline Computer Science \& Engineering\newline University at Buffalo, SUNY}}
%\address{123 Broadway}{City, State 12345}
\mobile{(716) 245 3011}
%\phone{(000) 111 1112}
%\fax{(000) 111 1113}
\email{swetankk@buffalo.edu}
\emailalt{swetank.saha@gmail.com}
%\homepage{swetank.in}
\cvsocial[github]{swetanksaha}
\cvsocial[linkedin]{swetanksaha}
\cvsocial[scholar]{CAt6EycAAAAJ}
% \photo[64pt][0pt]{me} % The first bracket is the picture height, the second is the thickness of the frame around the picture (0pt for no frame)
%\quote{"A witty and playful quotation" - John Smith}

\begin{document}

\makecvtitle
\vspace{-0.4in}
\section{Education}
\cvitem{2013-Present}{\textbf{Ph.D. Candidate} Computer Science \& Engineering | CGPA 3.81/4.0\newline
					  University at Buffalo (UB), SUNY, NY, US}
\cvitem{2009-2013}{\textbf{Bachelor of Technology} Computer Science \& Engineering (\textit{with Honors}) | CGPA 9.01/10.0\newline
            IIIT-Delhi, New Delhi, IN}
% \section{Research Interests}
% \cvlistitem{Wireless Networking}
% \cvlistitem{Mobile Systems}

% \section{Research Directions}
% \subsection{Current}
% \cvlistitem{Indoor 802.11ad/mmWave WLANs
% \begin{itemize}
% \renewcommand\labelitemi{--}
% \item Identify opportunities \& challenges in realizing 802.11ad indoor WLANs
% \item Design and evaluation of link adaptation mechanisms
% \item Leverage MPTCP to build fast \& reliable 802.11ad+ac network
% \item Identify energy consumption concerns in commercial 802.11ad clients
% \end{itemize}
% }
% \cvlistitem{LTE/WiFi coexistence in 5 GHz
% \begin{itemize}
% \renewcommand\labelitemi{--}
% \item Quantify coexistence issues with LTE-Unlicensed and Enterprise WLAN
% \item Propose standard-compliant WiFi-based solutions
% \end{itemize}
% }

% \subsection{Previous}
% \cvlistitem{WiFi power-performance tradeoffs in Smartphones
% \begin{itemize}
% \renewcommand\labelitemi{--}
% \item Identified the power-performance relationship in context of 802.11n/ac/ad in mobile devices
% \item Built accurate power models
% \end{itemize}
% }

\begin{comment}
\section{Publications}
\subsection{Conference}
\pub{C9}
{Fast and Infuriating: Performance and Pitfalls of 60 GHz WLANs Based on Consumer-Grade Hardware}
{Hany Assasa, Adrian Loch, Naveen Muralidhar Prakash, Roshan Shyamsunder Anantharamakrishna, Shivang Aggarwal, Daniel Steinmetzer, Dimitrios Koutsonikolas, Joerg Widmer, and Matthias Hollick}
{IEEE International Conference on Sensing, Communication and Networking}{SECON}{2018}

\pubsecond{C8}
{Medium Access and Transport Protocol Aspects in Practical 802.11ad Networks}
{Hany Assasa}{Adrian Loch, Dimitrios Koutsonikolas, Joerg Widmer}
{IEEE International Symposium on A World of Wireless, Mobile and Multimedia Networks}{WoWMoM}{2018}

\pub{C7}
{Multipath TCP in Smartphones: Impact on Performance, Energy, and CPU Utilization}
{Abhishek Kannan, Geunhyung Lee, Nishant Ravichandran, Parag Kamalakar Medhe, Naved Merchant, Dimitrios Koutsonikolas}
{ACM International Symposium on Mobility Management and Wireless Access}{MobiWac}{2017}

\pub{C6}
{A Detailed Look into Power Consumption of Commodity 60 GHz Devices}
{Tariq Siddiqui, Dimitrios Koutsonikolas, Adrian Loch, Joerg Widmer, Ramalingam Sridhar}
{IEEE International Symposium on A World of Wireless, Mobile and Multimedia Networks}{WoWMoM}{2017}

\pub{C5}
{A Feasibility Study of 60 GHz Indoor WLANs}
{Tariq Siddiqui, Viral Vijay Vira, Anuj Garg, Dimitrios Koutsonikolas}
{IEEE International Conference on Computer Communication and Networks}{ICCCN}{2016}

\pub{C4}
{Revisiting 802.11 Power Consumption Modeling in Smartphones}
{Pratham Malik, Selvaganesh Dharmeswaran, Dimitrios Koutsonikolas}
{IEEE International Symposium on A World of Wireless, Mobile and Multimedia Networks}{WoWMoM}{2016}

\pub{C3}
{Multi-Gigabit Indoor WLANs: Looking Beyond 2.4/5 GHz}
{Viral Vijay Vira, Anuj Garg, Dimitrios Koutsonikolas}
{IEEE International Conference on Communications}{ICC}{2016}

\pub{C2}
{Power-Throughput Tradeoffs of 802.11n/ac in Smartphones}
{Pratik Deshpande, Pranav P Inamdar, Ramanujan K Sheshadri, Dimitrios Koutsonikolas}
{IEEE Conference on Computer Communications}{INFOCOM}{2015}

\cvitem{C1}{\textbf{Take Control of Your SMSes : Designing an Usable Spam SMS Filtering System}\newline
           {\small Kuldeep Yadav, \underline{Swetank K Saha}, Ponnurangam Kumaraguru, Rohit Kumra}\newline
           {\small IEEE International Conference on Mobile Data Management (\textbf{MDM}) 2012}}

\subsection{Journal}
\pub{J2}
{X60: A Programmable Testbed for Wideband 60 GHz WLANs with Phased Arrays}
{Swetank Kumar Saha, Yasaman Ghasempour, Muhammad Kumail Haider, Tariq Siddiqui, Paulo De Melo, Neerad Somanchi, Luke Zakrajsek, Arjun Singh, Roshan Shyamsunder, Owen Torres, Daniel Uvaydov, Josep Miquel Jornet, Edward Knightly, Dimitrios Koutsonikolas, Dimitris Pados, Zhi Sun, Ngwe Thawdar}
{Elsevier Computer Communications}{COMCOM}{2018}

\pub{J1}
{60 GHz Indoor WLANs: Insights into Performance and Power Consumption}
{Darshan Godabanahal Malleshappa, Avinash Palamanda, Viral Vijay Vira, Anuj Garg, Dimitrios Koutsonikolas}
{Springer Wireless Networks}{WINE}{2017}

\subsection{Workshop}
\pub{W4}
{X60: A Programmable Testbed for Wideband 60 GHz WLANs with Phased Arrays}
{Yasaman Ghasempour, Muhammad Kumail Haider, Tariq Siddiqui, Paulo De Melo, Neerad Somanchi, Luke Zakrajsek, Arjun Singh, Owen Torres, Daniel Uvaydov, Josep Miquel Jornet, Edward Knightly, Dimitrios Koutsonikolas, Dimitris Pados, Zhi Sun}
{ACM Workshop on Wireless Network Testbeds, Experimental evaluation \& CHaracterization}{WiNTECH}{2017}

\pub{W3}
{Improving Connectivity, Coverage, and Capacity in 60 GHz Indoor WLANs Using Relays}
{Li Sun, Dimitrios Koutsonikolas}
{ACM Workshop on Wireless of the Students, by the Students, \& for the Students}{S$^3$}{2015}

\pub{W2}
{A First Look at TCP Performance in Indoor IEEE 802.11ad WLANs}
{Anuj Garg, Dimitrios Koutsonikolas}
{IEEE Conference on Computer Communications Workshops}{INFOCOM WKSHPS}{2015}

\pub{W1}
{On the Feasibility of Indoor IEEE 802.11ad WLANs}
{Viral Vijay Vira, Anuj Garg, Andrew Tennenbaum, Dimitrios Koutsonikolas}
{IEEE Conference on Computer Communications Workshops}{INFOCOM WKSHPS}{2015}

\subsection{Poster}
\pub{P4}
{Poster: AMuSe: An Agile Multipath TCP Scheduler for Dual-Band 802.11ad/ac Wireless LANs}
{Shivang Aggarwal, Dimitrios Koutsonikolas, Joerg Widmer}
{ACM International Conference on Mobile Computing and Networking}{MobiCom}{2018}

\pub{P3}
{Poster: Can MPTCP Improve Performance for Dual-Band 60 GHz/5 GHz Clients?}
{Roshan Shyamsunder, Naveen Muralidhar Prakash, Hany Assasa, Adrian Loch, Dimitrios Koutsonikolas, Joerg Widmer}
{ACM International Conference on Mobile Computing and Networking}{MobiCom}{2017}

\pub{P2}
{Poster: X60: A Programmable Testbed for Wideband 60 GHz WLANs with Phased Arrays}
{Yasaman Ghasempour, Muhammad Kumail Haider, Tariq Siddiqui, Paulo De Melo, Neerad Somanchi, Luke Zakrajsek, Arjun Singh, Owen Torres, Daniel Uvaydov, Josep Miquel Jornet, Edward Knightly, Dimitrios Koutsonikolas, Dimitris Pados, Zhi Sun}
{ACM International Conference on Mobile Computing and Networking}{MobiCom}{2017}

\pub{P1}
{LTE/WiFi Coexistence in 5 GHz: Bringing LTE-Awareness to Enterprise WiFi}
{Christina Vlachou, Kyu-Han Kim}
{Hewlett Packard Enterprise}{Technical Conference}{2017}

\subsection{Technical Report}
\pub{T1}
{60 GHz Multi-Gigabit Indoor WLANs: Dream or Reality?}
{Viral Vijay Vira, Anuj Garg, Dimitrios Koutsonikolas}
{arXiv:1509.04274}{arXiv}{2015}

\subsection{Under Submission/Review}
\cvitemwithcomment{IEEE TMC}{Fast and Infuriating: Performance and Pitfalls of 60 GHz WLANs Based on Consumer-Grade Hardware}{}
\cvitemwithcomment{Elsevier PMC}{Medium Access and Transport Protocol Aspects in Practical 802.11ad Networks}{}
\cvitemwithcomment{ACM Mobihoc 2019}{DeMiLTE: Detecting and Mitigating LTE Interference for Enterprise Wi-Fi in 5 GHz}{}
\cvitemwithcomment{}{AMuSe: An Agile Multipath TCP Scheduler for Dual-Band 802.11ad/ac Wireless LANs}{}
\end{comment}

\section{Technical Skills}
\cvitem{\textit{Languages}}{Proficient: C, Python | Intermediate: Java | Familiar: C++}
\cvitem{\textit{Linux Kernel}}{Wireless device drivers (\texttt{ath9k, ath10k, wil6210, iwlwifi}), TCP and MPTCP Networking subsystem}
%\cvitem{\textit{Simulators}}{ns2, ns3}
\cvitem{\textit{Networking}}{TCP/IP, HTTP, WiFi, LTE, packet sniffer, protocol analyzer}
\cvitem{\textit{Smartphone}}{Android applications (SDK/NDK), Platform (AOSP), Kernel}
%\cvitem{\textit{SDRs}}{USRP, GNURadio, LabView}
\cvitem{\textit{Web}}{Django, PHP, HTML, js}

\section{Work Experience}
\cventry{\textit{Summer 2017}\newline
\texttt{C}\newline
\texttt{Python}\newline
\texttt{OpenWRT}\newline
}{Research Associate (Intern)}{\textbf{Hewlett Packard Enterprise (HPE) Labs}}{Palo Alto, CA}{US}{
\underline{LTE-Unlicensed/WiFi Co-existence in 5 GHz for WiFi Access Points (APs)}
\vspace{0.025in}
\begin{itemize}
  \item Designed \textit{DeMiLTE}, the first WiFi (802.11ac)-based system for enterprise APs to detect, quantify, and react to LTE-U/LAA interference in real time, without requiring additional AP hardware
  \item Implemented the system inside the AP firmware making it light-weight and fully 802.11ac-standard compliant
  \item Improved AP downlink throughput by up to \textbf{110\%}, without requiring any client modifications
  \item Published 1 top-tier conference paper \hyperref[C3]{[C3]} and filed 1 patent \hyperref[P1]{[P1]}
\end{itemize}
}

\vspace{0.025in}
\cventry{\textit{Summer 2016}\newline
\texttt{C}\newline
\texttt{Python}\newline
\texttt{Drivers}\newline
\texttt{LEDE}\newline
}{Research Intern}{\textbf{IMDEA Networks Institute}}{Madrid}{Spain}{
\underline{Millimeter-wave Networking}
\vspace{0.025in}
\begin{itemize}
  \item Analyzed performance bottlenecks in next-generation of WiFi: 60 GHz (802.11ad)-based indoor WLANs
  \item Modified the Linux 802.11ad wireless device driver \texttt{(wil6210)} to export PHY/MAC information to userspace and allow userspace control over PHY parameters, like beam direction \link[\faGithub]{https://github.com/swetanksaha/wil6210}
  %\item Undertook an extensive measurement study with the instrumented APs and laptops to study 802.11ad links.
  \item Highlighted novel challenges and practical aspects like coverage and AP deployment, previously unreported
  \item Published the results of the study in a top-tier conference \hyperref[C2]{[C2]}
\end{itemize}
}

\vspace{0.025in}
\cventry{\textit{Summer 2013}\newline
\texttt{Python}\newline
\texttt{Java}\newline
\texttt{Android}\newline
\texttt{Django}\newline
}{Software Developer}{\textbf{Google Summer of Code 2013}}{Google}{}{
\underline{\link[Funf]{https://www.funf.org/}: Open sensing and data collection framework for Android (acquired by Google)} \link[\faGithub]{https://github.com/funf-org/funf-inabox}
\vspace{0.025in}
\begin{itemize}
  \item Improved Funf-in-a-box (FIAB), a service for users to build custom data collection app with zero programming
  \item Ported the entire FIAB service from an always-on architecture running on a single EC2 server to on-demand VM instantiation on Google cloud (in <1 month), significantly reducing costs \& increasing performance. \link[\faExternalLink]{https://opensource.googleblog.com/2013/08/who-is-new-in-google-summer-of-code_30.html}
  \item Added support for configuring and deploying custom surveys and capturing additional user input
\end{itemize}
}

\vspace{0.025in}
\cventry{\textit{Summer 2012}\newline
\texttt{Android}\newline
\texttt{Ubuntu Juju}\newline}{Research Intern}{\textbf{Airbus India, EADS Innovation Works}}{Bangalore}{India}{
\underline{Image based localization techniques}
\vspace{0.025in}
\begin{itemize}
  \item Built an indoor localization service that uses image features to estimate location of an Android device
  \item Implemented it as a distributed system that offloads image processing to a remote server
  \item Packaged and deployed the system as a Juju charm for easy production orchestration
\end{itemize}
}

% \cventry{\textit{2013--Present}}{Research Assistant (RA) | Teaching Assistant (TA)}{\textbf{University at Buffalo}, SUNY}{NY}{US}{}

%\cventry{\textit{2013--Present}}{Teaching Assistant}{University at Buffalo, SUNY}{NY}{US}{}
%Designed and developed assignments and automated grader for the Modern Networking Concepts course\newline\httplink[https://cse4589.github.io/]{https://cse4589.github.io/} [\link[Github Source]{https://github.com/cse4589}]


\section{Projects}
\cventry{\texttt{C}\newline
\texttt{Python}\newline
\texttt{TCP}\newline
\texttt{MPTCP}\newline
\texttt{tc}\newline
\texttt{Kernel}\newline
}{Multipath TCP (MPTCP) for Dual-band WiGig+WiFi networks}{NSF Funded}{}{}{
\begin{itemize}
  \item Leveraged MPTCP to engage two network interfaces simultaneously to acheive throughput of $\sim$2.2 Gbps
  \item Instrumented MPTCP (Linux 4.x) using \texttt{kernel probes} to monitor over 32 parameters in real-time
  %\item Instrumented TCP/MPTCP send and recv queues to track input and output timestamp for each byte.
  \item Designed \underline{AMuSe}, a new MPTCP packet scheduler, that improved overall throughput by 2.5x
  \item Implemented it as a kernel module replacing the default MPTCP scheduler
  % \item Evaluated AMuSe extensively which showed up to \textbf{2.5x} throughput improvement over the default scheduler
  \item Submitted a top-tier conference paper (under review)
\end{itemize}
}

\cventry{\texttt{Python}\newline
\texttt{C}\newline
\texttt{Drivers}\newline
\texttt{Android}\newline
}{Power-Performance Tradeoffs for Mobile Devices in Next Generation WiFi Networks}{NSF Funded}{}{}{
\begin{itemize}
  \item Developed a set of tools to enable large-scale performance and power measurement of mobile devices involving data collection \& sync across heterogeneous systems: Linux (Wireless AP), Android, and Power monitor
  \item Modified the Linux wireless drivers (\texttt{ath9k}, \texttt{ath10k}) to expose userspace control of several PHY/MAC parameters.
  \item Analyzed the power-performance tradeoff of both the NIC and CPU for uplink/downlink data transfer.
  \item Improved the accuracy of the state-of-the-art power model by up to \textbf{40\%}.
  \item Pubished 1 top-tier conference paper \hyperref[C1]{[C1]}.
\end{itemize}
}

\cventry{\texttt{Labview}\newline
\texttt{Python}\newline
}{X60: A highly Re-configurable Multi-Gigabit Testbed for 60 GHz research}{NSF Funded}{}{}{
\begin{itemize}
  \item Set up the first ever software-defined 60 GHz testbed that offers configurability at PHY/MAC/Network layers.
  \item Added several components (e.g., AGC) to existing NI codebase to enable realistic measurements.
  \item Developed set of tools to automate the measurement cycle reducing time from hours to several minutes.
  \item Exposed user controls to enable measurements by external researchers (used by Rice University \& UT Austin).
  \item Published 1 journal paper \hyperref[J1]{[J1]}.
\end{itemize}
}

\cventry{\texttt{C}\newline
\texttt{Python}\newline
\texttt{expect}\newline
\texttt{select}\newline
\texttt{Socket API}\newline
}{Programming Assignments and Automated Grader for the Computer Networking course}{}{}{\link[\faGithub]{https://cse4589.github.io}}{
\begin{itemize}
  \item Designed and developed a set of three networking programming assignments (PAs), now standard in the CSE department at UB for the networking course.
  \item Developed the auto-grader as a distributed application that fully automates the entire process of packaging, uploading and grading student submissions over a real network of five servers.
  \item PA 1: Text Chat \& File Transfer Application \link[\faExternalLink]{https://cse4589.github.io/pa1/} (Source: \link[\faGithub]{https://github.com/swetanksaha/cse4589-pa1})
        \begin{itemize}
          \item Introduces basic socket programming, client-server model and use of \texttt{select} syscall.
          \item Test cases check understanding of the byte-stream model of TCP (vs. UDP's packet-based).
          %\item Bonus file-transfer components allows students to compare file I/O vs. network I/O.
        \end{itemize}
  \item PA 2: Reliable Transport Protocols \link[\faExternalLink]{https://cse4589.github.io/pa2/} (Source: \link[\faGithub]{https://github.com/swetanksaha/cse4589-pa2})
        \begin{itemize}
          \item Uses a simulated environment to ask students to implement three reliable data transport protocols.
          \item Requires implementation of multiple alarms with a single simulator timer.
          \item Modified an existing simulator to add sanity checks and internal support for different test cases.
        \end{itemize}
  \item PA 3: Software Defined Routing \link[\faExternalLink]{https://cse4589.github.io/pa3/} (Source: \link[\faGithub]{https://github.com/swetanksaha/cse4589-pa3})
        \begin{itemize}
          \item Involves implementation of a simplified version of the distance vector routing protocol (using UDP).
          \item Introduces students to the concept of control and data planes asking them to implement both functionalities.
          \item Requires contruction of raw packets in code using techniques akin to the Linux kernel (Sample code: \link[\faGithub]{https://github.com/swetanksaha/cse4589-pa3/blob/master/SampleCode/swetankk/src/control_header_lib.c}).
        \end{itemize}
\end{itemize}
}


\section{Selected Publications \& Patents}
\pub{C3}
{DeMiLTE: Detecting and Mitigating LTE Interference for Enterprise Wi-Fi in 5 GHz}
{ACM International Symposium on Mobile Ad Hoc Networking and Computing}{MobiHoc}{2019}{}

\pub{C2}
{Fast and Infuriating: Performance and Pitfalls of 60 GHz WLANs Based on Consumer Hardware}
{IEEE International Conference on Sensing, Communication and Networking}{SECON}{2018}{https://ieeexplore.ieee.org/document/8397123}

\pub{C1}
{Power-throughput tradeoffs of 802.11n/ac in smartphones}
{IEEE Conference on Computer Communications}{INFOCOM}{2015}{https://ieeexplore.ieee.org/document/7218372}

\pub{J1}
{X60: A Programmable Testbed for Wideband 60 GHz WLANs with Phased Arrays}
{Elsevier Computer Communications}{COMCOM}{2019}{https://doi.org/10.1016/j.comcom.2018.09.005}

\pub{P1}
{LTE Interference Detection and Mitigation for Wi-Fi Links}
{US Patent}{Record ID: 90547646}{Application \#: 15/962,722}{}

\section{Selected Honors \& Awards}
\cvlistitem{\textbf{2$^{nd}$ Runner Up} ACM Student Research Competition (SRC) 2017}
\cvlistitem{\textbf{Best Paper Runner Up} ACM WiNTECH 2017}
\cvlistitem{\textbf{1$^{st}$} prize at the UB School of Engineering \& Applied Sciences (SEAS) Lightning Talk Competition}
\cvlistitem{\textbf{Winner} of the \textit{Tally Innovation Award} at the \link[All-India Jedi Project Challenge 2012]{https://jed-i.in/challenge/}, IISc., Bangalore.}
\cvlistitem{Travel Grants: IEEE SECON (2018), ACM MobiCom (2017, 2018), ACM IMC (2015)}

\section{Professional Service}
\cvitem{Chair}{ACM Wireless of the Students, by the Students, and for the Students (S$^3$) Workshop 2018}
\cvitem{Journal Reviewer}{2019: IEEE ToN, IEEE VTM | 2017: IEEE TMC, IEEE TWC}
% \cvitem{Invited Journal Reviews}{
% \begin{itemize}
% \item IEEE Transactions on Communications 2018
% \item IEEE Transactions on Mobile Computing (TMC) 2017
% \item MDPI Applied Sciences 2017
% \item IEEE Symposium on Computers and Communications (ISCC) 2017
% \item IEEE Transactions on Wireless Communications (TWC) 2017
% \item MDPI Sensors 2015
% \end{itemize}
% }
% \cvitem{External Reviewer}{
% \begin{itemize}
% \item IEEE Conference on Computer Communications (INFOCOM) 2016, 2015
% \item IEEE International Symposium on A World of Wireless, Mobile and Multimedia Networks (WoWMoM) 2016, 2015
% \item IEEE International Conference on Sensing, Communication and Networking (SECON) 2016, 2015
% \end{itemize}
% }

\end{document}
